\documentclass[fleqn]{article}                     % onecolumn (standard format)

%\linespread{1.5}
\usepackage{amsmath}
\usepackage{wasysym}
\usepackage{graphicx}
\usepackage{fdsymbol}
\usepackage{pdfpages}
\usepackage{natbib}
\usepackage{xcolor}
\bibliographystyle{spbasic}
\usepackage[margin=1.25in]{geometry}
\usepackage[modulo]{lineno}
%\linenumbers

\definecolor{darkmagenta}{rgb}{0.9, 0.0, 0.9}
\newcommand{\comment}[3][darkmagenta]{\textcolor{#1}{\textbf{#2}: #3}}
\newcommand{\LiNA}{\textbf{LiNA} }


%% to convert into bw pdf, run this in the terminal
%gs  -o grayscale.pdf  -sDEVICE=pdfwrite  -sColorConversionStrategy=Gray  -sProcessColorModel=DeviceGray   draft1.pdf

%% to spellcheck using aspell
% aspell --lang=en --mode=tex check draft4.tex

\title{LiNA: A Light-Nutrient-Aggregation plankton model}
\begin{document}

\maketitle
%%%%%%%%%%%%%%%%%%%%%%%%%%%%%%%%%%%%%%%%%%%%%%%%%%%%%%%%%%%%%%%%%%%%%%%%%%%%%%%%
%%%%%%%%%%%%%%%%%%%%%%%%%%%%%%%%%%%%%%%%%%%%%%%%%%%%%%%%%%%%%%%%%%%%%%%%%%%%%%%%
%\begin{abstract}
%There is no abstract yet.
%\end{abstract}

%%%%%%%%%%%%%%%%%%%%%%%%%%%%%%%%%%%%%%%%%%%%%%%%%%%%%%%%%%%%%%%%%%%%%%%%%%%%%%%%
%%%%%%%%%%%%%%%%%%%%%%%%%%%%%%%%%%%%%%%%%%%%%%%%%%%%%%%%%%%%%%%%%%%%%%%%%%%%%%%%

\section{Model description}

Modular by design, \LiNA is a model created to provide a simulation framework to explore the feedback among \textbf{Li}ght, \textbf{N}utrients and \textbf{A}ggregation in the pelagic.
\LiNA consists in two independent but interconnected parts: (1) light attenuation created by suspended particular matter --SPM, loosely the sum of suspended plankton, aggregates and lithogenic material--, and (2) nutrient circulation created by nutrient sequestration and leakage by plankton, aggregates and detritus.
Such processes are subject to physical forcing provided by an external physical driver.

\comment{og}{Do we need a differentiation between labile and refractory detritus?}

\LiNA manages seven state variables (Table~\ref{tab:stateVariables}): four concentrations --phytoplankton, aggregates,free exopolysaccharides concentration, lithogenic material and detritus-- and two nutrients --Nitrogen and Phosphorus-- .
The nutrient-plankton-detritus dynamic is calculated by a simple \textbf{NPZD} model, but enhanced by a detailed physiology of nutrient uptake, light harvesting and Exopolysaccharides (EPS) production.
The aggregate-lithogenic material dynamic is described by the \textbf{Agg} model with the addition of nutrient composition.
\comment{og}{An adaptive scheme for plankton and aggregate size can be added... probably not now.}

\begin{table}[h!]
  \centering
  \caption{\LiNA state variables}
  \begin{tabular}{cll}
  \hline
    \textbf{Symbol} & \textbf{Name} & \textbf{Unit}\\ \hline
    $X$ & Phytoplankton concentration & \\
    $A$ & Aggregate concentration & \\
    $E$ & Free exopolysaccharides concentration & \\
    $L$ & Primary particles, lithogenic material & \\
    $D$ & Detritus & \\
    $N$ & Nitrogen concentration & \\
    $P$ & Phosphorus concentration & \\
    
    \hline
    $Q^A_N$ &Aggregate Nitrogen to Carbon ratio \\
    $Q^A_P$ \\
    $Q^D_N$ &Detritus Nitrogen to Carbon ratio\\
    $Q^D_P$\\
    $Q^X_N$&Phytoplankton Nitrogen to Carbon ratio\\
    $Q^X_P$\\
    $\psi$\\
  \end{tabular}
  \label{tab:stateVariables}
\end{table}


%%%%%%%%%%%%%%%%%%%%%%%%%%%%%%%%%%%%%%%%%%%%%%%%%%%%%%%%%%%%%%%%%%%%%%%%%%%%%%%%
%%%%%%%%%%%%%%%%%%%%%%%%%%%%%%%%%%%%%%%%%%%%%%%%%%%%%%%%%%%%%%%%%%%%%%%%%%%%%%%%
\section{\LiNA model equations}


\subsection{Ecological state variables}
The functions for $X$, $N$ and $D$ dynamics are managed externally by \textbf{NPZD} model
The sinking velocities should be implemented in the $\_GET\_VERTICAL\_MOVEMENT\_$ function, so they were removed from these equations.

\begin{eqnarray}
  \frac{dX}{dt} &=&  (\mu_X - m_X- w_X - C_X)\  X \\
  %\frac{dA}{dt} &=&  C_{Tot} - (w_A + m_A + k_B) \ A\\
  \frac{dA}{dt} &=&  C_{Tot} - (m_A + k_B) \ A\\
  \frac{dE}{dt} &=& \eta \cdot X - h\cdot A \\
  \frac{dL}{dt} &=& r_L - C_L\ L + k_B\ \psi\ A \\
  \frac{dD}{dt} &=& m_X \  X - (m_D + C_D)\ D + k_B\ (1-\psi)\ A \\
  \frac{dN}{dt} &=& -\gamma_N \ X + m_D\ D \  Q^D_N + m_A\ A\  Q^A_N + e_N\\
  \frac{dP}{dt} &=& -\gamma_P\  X + m_D\ D \  Q^D_P + m_A\ A\  Q^A_P + e_P
\end{eqnarray}


We allow the physical advection scheme to alter the nutrient content by dilution of external sources, by setting the FABM parameter ($no_river_dilution=.false.$).

\subsection{Phytoplankton physiology}
\comment{og}{This part is not included in the simple \textbf{NPZD}}

Internal nutrient quota change
\begin{eqnarray}
  \frac{dQ^X_N}{dt} &=&  \gamma_N - \mu_X \  Q^X_N \\
  \frac{dQ^X_P}{dt} &=&  \gamma_P- \mu_X \  Q^X_P
\end{eqnarray}

Normalized nutrient quota
\begin{equation}
  q_N = \frac{Q^X_N-Q^0_N}{Q^*_N-Q^0_N};\ \ q_P = \frac{Q^X_P-Q^0_P}{Q^*_P-Q^0_P}
\end{equation}

Nutrient uptake
\begin{equation}
  \gamma_N = \gamma^*_N \frac{N \cdot A_N}{\gamma^*_N+ N \cdot A_N} \cdot (1-q_N)
\end{equation}

\begin{equation}
  \gamma_P = \gamma^*_P \frac{P \cdot A_P}{\gamma^*_P + P \cdot A_P} \cdot (1-q_P)
\end{equation}

Phytoplankton growth rate
\begin{equation}
  \mu_X= c_I\  c\  \mu_{\text{max}} - R
\end{equation}

Phytoplankton respiration rate
\begin{equation}
  R = \zeta \cdot \gamma_N
\end{equation}

Light limitation coefficient
\begin{equation}
  c_I(I) = 1-\textrm{e}^{-\alpha I/ \mu_{\text{max}}}
\end{equation}

EPS production (this is the old formulation, NOT USED ANYMORE)
\begin{equation*}
  \eta = E_{min} + (E_{\textrm{max}}-E_{\textrm{min}})(1+0.5\ \tanh(B^*\dot{c}-B))
\end{equation*}

THIS IS THE NEW FORMULATION DERIVATED FROM OPTIMALITY.
It is already implemented in the LiNA code.
\begin{equation}
  \eta = E_{\textrm{min}} + (E_{\textrm{max}}-E_{\textrm{min}})(\mu_X-\gamma_{\lim}/(1-g_n(q_{\lim}))*g'_n(q_{\lim})
\end{equation}

Nutrient limitation coefficient
\begin{equation}
  c = \mathcal{C}_n(q_N,q_P)
\end{equation}

Co-limitation function (\comment{og}{please make this a function, it could change in the future})
\begin{equation}
  \mathcal{C}_n(q_i, q_j) = q_i \cdot g_n(qj/qi) \cdot (1 + q_i\ q_j\ MI + c_n)
\end{equation}

Queuing theory limitation
\begin{equation}
  g_n(r) = \frac{r - r^{1+n}}{1 - r^{1+n}}
\end{equation}

Nutritional stress
\begin{equation}
  \dot{c} = \frac{\partial c}{\partial qN}\frac{dq_N}{dt}+\frac{\partial c}{\partial q_P}\frac{dq_P}{dt}
\end{equation}

Independence coefficient
\begin{equation}
  MI = MI^*(1+q_N)
\end{equation}

Nutritional state modified plankton density
\begin{equation}
  \rho_X = \rho^*_X\cdot d_X^{-a_\rho}\cdot(1-(1-d_X^{-a_\rho})\cdot c_I\cdot  c)
\end{equation}

Sinking velocity
\begin{equation}
  w_X = \frac{1}{18\ \mu_w} (\rho_X-\rho_w)\ g\ d_X^2
\end{equation}

\subsection{Aggregation processes and composition}
\comment{og}{This part is included in \textbf{Agg}}

\begin{eqnarray}
  \frac{d\psi}{dt} &=& C_L \frac{L}{A} - k_B\ \psi\ -\frac{\psi}{A}\ \frac{dA}{dt} \\
  \frac{dQ^A_N}{dt} &=& C_X\ X\ Q^X_N +  C_D\ D\ Q^D_N -(w_A+m_A+k_B)\ A\ Q^A_N\\
  \frac{dQ^A_P}{dt} &=& C_X\ X\ Q^X_P +  C_D\ D\ Q^D_P -(w_A+m_A+k_B)\ A\ Q^A_P
\end{eqnarray}

$\psi$ is the lithogenous fraction in the aggregate.
$\phi$ is the porosity.

\begin{equation}
  C_X = k_C\ \left(V_A+\frac{X}{\rho_X}\right)
\end{equation}

\begin{equation}
  C_L = k_C\ \left(V_A+\frac{L}{\rho_L}\right)
\end{equation}

\begin{equation}
  C_D = k_C\ \left(V_A+\frac{D}{\rho_D}\right)
\end{equation}

\begin{equation}
  d_A = \kappa_0 + \kappa_1\ (A/\sqrt{G})
\end{equation}

$\kappa_0$ and $\kappa_1$ are constants

\begin{equation}
  C_{Tot} = C_X\ X + C_L\ L + C_D\ D
\end{equation}

\begin{equation}
  k_C = \alpha'\ G \ f_C
\end{equation}

\begin{equation}
  k_B = \beta'\ G^{3/2}\ d_A^2 + \textrm{some exponential decayment}
\end{equation}

\begin{equation}
  f_C = \frac{E}{E+E^*}
\end{equation}

\begin{equation}
  V_A = A\  \frac{\psi/ \rho_L + (1-\psi)/ \rho_D}{1-\phi}
\end{equation}

\begin{equation}
  w_A = \frac{1}{18\ \mu_w} (\rho_A-\rho_w)\ g\ d_A^2
\end{equation}

Sinking is calculated in the  get vertical movement part in the Agg model.
Resuspension can be included in the SPM model.

\begin{equation}
  \rho_A = \frac{A}{V_A}+\phi\ \rho_w
\end{equation}

\begin{equation}
r_L = \frac{E_0}{z}\max \left( (\frac{\tau}{\tau_C}-1) ,0 \right)
\end{equation}

$\tau_C$ is the critical shear stress.

\subsection{Detritus composition}
\comment{og}{This part is not included in the simple \textbf{NPZD}}

\begin{eqnarray}
  \frac{dQ^D_N}{dt} &=&  m_X\ X\ Q^X_N - (m_D + w_D + C_D)\ D\ Q^D_N + k_B\ (1-\psi)\ A\ Q^A_N \\
  \frac{dQ^D_P}{dt} &=& m_X\ X\ Q^X_P - (m_D + w_D + C_D)\ D\ Q^D_P + k_B\ (1-\psi)\ A\ Q^A_P
\end{eqnarray}

\subsection{Light atenuation}
Attenuation coefficient
\comment{og}{This is not settled yet.}
\begin{equation}
  k = k'\cdot \textrm{Chla} + k'' \cdot \textrm{SPM}
\end{equation}

\begin{table*}[h]
  \centering
  \caption{Inner state variables and model parameters}
  \begin{tabular}{clll}
  \hline
    \textbf{Symbol} & \textbf{Name} & \textbf{Unit}& is Parameter \\ \hline
    & \emph{Plankton compartment} & \\
    $A_N$&&&Y\\
    $A_P$&&&Y\\
    $e_N$ & external Nitrogen input&Y\\
    $e_P$ & external Phosphorus input&Y \\
    $Q^X_N$ & Nitrogen to carbon ratio in phytoplankton & mol-N mol-C$^{-1}$ \\
    $Q^X_P$ & Phosphorus to carbon ratio in phytoplankton & mol-P mol-C$^{-1}$ \\
    $Q^*_N$ & Maximum nitrogen to carbon ratio & mol-N mol-C$^{-1}$ &Y\\
    $Q^*_P$ & Maximum phosphorus to carbon ratio & mol-P mol-C$^{-1}$ &Y\\
    $Q^0_N$ & Baseline nitrogen to carbon ratio & mol-N mol-C$^{-1}$ &Y\\
    $Q^0_P$ & Baseline phosphorus to carbon ratio & mol-P mol-C$^{-1}$ &Y\\
    $h$ & &&Y\\
    $d_X$&Phytoplankton Size& m &\\
    $d_A$&Aggregate Size& m &\\
    $\rho_L$& Density of Lithogeneous & kg m-3 & Y\\
    $\rho_A$& Density of Aggregates & kg m-3&\\
    $\rho_X$& Density of Phytoplankton & kg m-3 &\\
     $\rho^*_X$& Base Density of Phytoplankton & kg m-3 &Y\\
    $\rho_w$&Density of Water & kg m-3&Y\\
    $a_\rho$&&&Y\\
    $m_D$&\\
    $w_D$&\\
    $C_D$&\\
    $\mu_X$ & Phytoplankton growth rate & d$^{-1}$ \\
    $m_X$ & Phytoplankton mortality rate & d$^{-1}$ \\
    $w_X$ & Phytoplankton sinking rate & d$^{-1}$ \\
    $C_X$ & Phytoplankton coagulation rate & d$^{-1}$ \\
    $\gamma_N$ & Nitrogen uptake rate & mol-N mol-C$^{-1}$ d$^{-1}$ \\
    $\gamma_P$ & Phosphorus uptake rate & mol-P mol-C$^{-1}$ d$^{-1}$ \\
    $\gamma_N*$ & Maximum Nitrogen uptake rate & mol-N mol-C$^{-1}$ d$^{-1}$&Y \\
    $\gamma_P*$ & Maximum Phosphorus uptake rate & mol-P mol-C$^{-1}$ d$^{-1}$&Y \\
    $c_I$ & Light dependence in growth rate & -- \\
    $c$ & Physiology colimitation in growth rate & -- \\
    $MI$ & Metabolic independence & -- \\
    $\alpha$ & Photosynthesis slope & &Y \\
    $\mu_{\textrm{max}}$ & Maximum growth rate & d$^{-1}$ &Y\\
    $\mu_w$&Water resistance?&&Y\\
    $E$ & Exopolymer material (EPS) production & mol-C m$^{-3}$ \\
    $E_{\textrm{min}}$ & Minimum EPS production & mol-C m$^{-3}$&Y \\
    $E_{\textrm{max}}$ & Maximum EPS production & mol-C m$^{-3}$ &Y\\
    $\dot{c}$ & Physiology stress for nutrient limitation & d$^{-1}$ \\
    $B$ & Parameter for EPS production & --&Y \\
    $B^*$ & Other parameter for EPS production & d &Y\\
    $\zeta$ & Maximum/Basline respiration rate &&Y\\
    $c_n$&Colimitaiton parameter&&Y\\
    $MI^*$ & Intrinsec metabolic independence & --&Y \\ \hline
    & \emph{Aggregate compartment} & \\
    $w_a$&Sinkingspeed of aggregates&ms-1&N\\
    $m_a$\\
    $\psi$ &&\\
    $g$&Gravitation acceleration&$m2s2$&Y\\
    $C_L$ &&\\
    $C_{tot}$&\\
    $k_B$&&\\
    $V_A$ & Volume of Aggregate & m-3 \\
    $Q^A_N$ & Nitrogen to carbon ratio in aggregate & mol-N mol-C$^{-1}$ \\
    $Q^A_P$ & Phosphorus to carbon ratio in aggregate & mol-P mol-C$^{-1}$ \\
    \hline

  \end{tabular}
\end{table*}





\end{document}

%%%%%%%%%%%%%%%%%%%%%%%%%
%%%%%%%%%%%%%%%%%%%%%%%%%
%%%%%%%%%%%%%%%%%%%%%%%%%
%%%%%%%%%%%%%%%%%%%%%%%%%
%%%%%%%%%%%%%%%%%%%%%%%%%
%%%%%%%%%%%%%%%%%%%%%%%%%
